\documentclass[]{beamer}

\usepackage[british,english,ngerman]{babel}
\usepackage[utf8]{inputenc}

\usecolortheme{beaver}

%%%%%%%%%%%%%%%%%%%%%%%%%%%%%%%%%%%%%%%%%%%%%%%%%%%%%%%%%%%%%%%
% Titlepage 
%%%%%%%%%%%%%%%%%%%%%%%%%%%%%%%%%%%%%%%%%%%%%%%%%%%%%%%%%%%%%%%
\title[Short version]{Probabilistic Scene Understanding using\\ Virtual Reality and Markov Logic Networks}
\subtitle[]{}
\date[]{18.12.2018}
\author[D. Dieckmann]{Dominik Dieckmann}
\institute[Uni Bremen]{Institute for Artificial Intelligence \\ University Bremen}
%\logo{\includegraphics[scale=.1]{unihb/unilogo-transp.pdf} \includegraphics[scale=.1]{unihb/logo-ai.pdf}}
%%%%%%%%%%%%%%%%%%%%%%%%%%%%%%%%%%%%%%%%%%%%%%%

\begin{document}
\beamertemplatenavigationsymbolsempty

\begin{frame}
	\maketitle
\end{frame}

\begin{frame}{Autonomous Robots in Houshold Environments}
	\begin{itemize}
		\item perception component
			\begin{itemize}
				\item detect objects
				\item analyse objects
			\end{itemize}
		\item reasoning component
			\begin{itemize}
				\item identify/classify objects based on their visual cues
				\item needs to be trained
			\end{itemize}
	\end{itemize}
\end{frame}

\begin{frame}{Creation of Training Data}
time and resource intensive:
	\begin{itemize}
		\item manually creating scenarios and images
		\item no groundtruth
	\end{itemize}
	
$\rightarrow$ synthetic images from a game engine
\end{frame}

\begin{frame}{System Setup}
	\begin{itemize}
		\item list of objects, classes and scenarios
		\item \textsc{Unreal Engine} to create \textit{Unreal Images}
		\item \textsc{RoboSherlock} analysis the images
		\item learn a \textit{Markov Logic Network} 
		\item classify the objects in the images
	\end{itemize}
\end{frame}

\begin{frame}{Unreal Engine}
	\begin{itemize}
		\item photorealism
		\item rendering in realtime
		\item open source
	\end{itemize}
\end{frame}

\begin{frame}{Unreal Images}
	\begin{itemize}
		\item Assets
			\begin{itemize}
				\item scanned 3D-models of the objects
				\item kitchen environment
			\end{itemize}
		\item URoboVision plugin
			\begin{itemize}
				\item create RGBD image from a scene
				\item create \textit{ObjectImage} and  \textit{ObjectMap} 
				\item send them to \textsc{RoboSherlock}
			\end{itemize}
		\item RSpawnBox class
			\begin{itemize}
				\item visual representation of scene space
				\item rotates camera around the scene
			\end{itemize}
	\end{itemize}
\end{frame}

\begin{frame}{Unreal Images}
	\begin{itemize}
		\item 114 scenes
		\item 2 - 5 objects per scene
		\item 5 viewpoints per scene
		\item only objects of one scenario per scene
		\item total of 570 images
	\end{itemize}
\end{frame}


\begin{frame}{RoboSherlock}
	\begin{itemize}
		\item based on \textit{UIMA}
		\item segments images and creates object hypotheses
		\item annotates attributes of the hypotheses
	\end{itemize}
\end{frame}

\begin{frame}{Perceptionpipeline}
Annotates the following attributes:
	\begin{itemize}
		\item color
		\item size
		\item shape
		\item goggles\_\{Logo, Text, Product\}
		\item instance
		\item object
	\end{itemize}
\end{frame}

\begin{frame}{UnrealGTAnnotator}
Annotates the groundtruth by:
	\begin{itemize}
		\item counting the color of pixels in the ObjectImage
		\item looking up the corresponding asset 
		\item looking up the groundtruth for that asset
		\item setting the groundtruth 
	\end{itemize}

\end{frame}


\begin{frame}{Markov Logic Networks}
what are they?\\
advantages for object classification
\end{frame}

\begin{frame}{Experiments}

baseline: PR2 paper

show results...

\end{frame}


\begin{frame}{Conclusion}
it just works
\end{frame}

\end{document}

