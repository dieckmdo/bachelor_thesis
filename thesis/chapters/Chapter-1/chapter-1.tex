\graphicspath{{./images/}}      
\def\CHAPTERONE{./chapters/Chapter-1} 

\chapter{Einleitung}
\label{chap:introduction}
%	\input{\CHAPTERONE /motivation}
In den letzten Jahren haben autonome Roboter für Haushaltsumgebungen einen hohen Stellenwert in der Forschung eingenommen \todo{citation ?, sonst anders formulieren}. Ob es darum geht Essen zu kochen oder aufzuräumen, die möglichen Aufgaben sind nahezu unendlich und erleichtern so das Leben. Gerade in einer Gesellschaft in der die Bevölkerung immer älter wird \cite{peopleGetOlder} ist Hilfe durch autonome Roboter eine erstrebenswerte Zukunft. \par
Die für uns teils simpel erscheinenden Aufgaben, wie das einräumen einen Schranks, sind für einen Roboter schon komplexe Vorgänge. Er muss nicht nur die Objekte, die er einräumen soll, eindeutig identifizieren, sondern diese auch richtig und den äußerlichen Eigenschaften entsprechend greifen. So fässt man eine Packung Mehl anders an als eine Flasche Saft. All das muss der Roboter aus den Informationen, die seine Sensoren und Kameras erfassen, Schlussfolgern, um so die eigentliche Aufgabe, das Einräumen des Schranks, überhaupt beginnen zu können. \par
Umgebungen, die von Menschen bewohnt werden, sind jedoch starken nichtdeterministischen Schwankungen unterworfen. Es werden Objekte verrückt oder komplett aus dem Raum entfernt, sodass selbst wiederholt auszuführende Aufgaben niemals genau gleich ablaufen. Das erfordert von dem Roboter ein hohes Maß an Robustheit gegenüber Veränderungen seiner Umwelt. Grundlage um auf Veränderungen reagieren zu können ist dabei die Wahrnehmung oder Perzeption des Roboters. Über Kameras und Sensoren werden Daten über die Umgebung aufgenommen und dem Roboter so ermöglicht die verrückten Objekte aufzufinden. \par 
Allerdings kann der Roboter aus den rohen Sensor- und Kameradaten keine Annahme darüber machen, wo sich überhaupt Objekte befinden. Es müssen also spezielle Perzeptionsalgorithmen zum Einsatz kommen, die die Daten verarbeiten und interpretieren. So liegen dem Roboter Objekthypothesen über den potenziellen Standort von Objekten vor, sowie Informationen über die einzelnen Objekthypothesen. Mit all diesem Wissen über die Objekte kann der Roboter jedoch nichts anfangen, wenn er nicht eine Möglichkeit besitzt aus den Informationen abzuleiten, um was für Objekte es sich genau handelt. Dazu werden in einer \gls{kb} die Informationen zusammengeführt und verknüpft, sodass der Roboter aus bestimmten Informationen über eine Objekthypothese schlussfolgern kann, um was für ein Objekt es sich handelt.    \par    
Das Schlussfolgern ist also eine Zentrale Sache für das Bestehen des Roboters in der veränderlichen Haushaltsumgebung. Das trainieren solcher \glspl{kb} ist jedoch zeit- und ressourcenintensiv, denn das Trainieren erfordert es nicht nur, Beispiele einer wahrnehmbaren Umgebung zu erstellen und den Roboter daraus die Objektinformationen ziehen zu lassen, sondern für optimale Ergebnisse, wird auch eine große Menge von Trainingsdaten benötigt. Das Trainieren  einer guten \gls{kb} ist damit zwar erstrebenswert aber höchst unattraktiv. \par      
Abhilfe könnten hier nun moderne \glspl{gameengine} schaffen. Diese bieten die Möglichkeit fotorealistische Umgebungen in Echtzeit zu rendern und sich in dieser virtuellen Realität auch zu bewegen. Es bietet sich also die Möglichkeit Trainingsdaten in der virtuellen Realität aufzunehmen und damit eine \gls{kb} zu trainieren. \par  
Ziel dieser Arbeit ist es nun, zu testen, ob man mit Bildern aus einer Game Engine, eine \gls{kb} antrainieren kann und das folgende Schlussfolgern mit der \gls{kb} Ergebnisse liefert, die mit einer mit echten Bilden trainierten \gls{kb} vergleichbar sind. Das \gls{iai}\footnote{\url{http://ai.uni-bremen.de/}} der Universität Bremen hat die Küche, in der mit einem PR2-Roboter geforscht wird, möglichst realistisch in die Unreal Engine (siehe Kapitel \ref{sec:unrealengine} auf S.\pageref{sec:unrealengine}) übertragen. Hier können nun Szenen erzeugt und dann in ein Robotersystem eingespeist werden. Das vom IAI entwickelte \gls{framework} \robosherlock (siehe Kapitel \ref{sec:robosherlock} auf S.\pageref{sec:robosherlock}), wird dann Perzeptionsalgorithmen auf den Bildern laufen lassen und so Informationen über die gewünschten Objekte herausziehen. Die Informationen über die Objekte basieren auf der Theorie der Attribut-basierten Objekterkennung. Dabei werden Objekten Eigenschaften wie Farbe, Form und Größe zugewiesen, um sie voneinander zu unterscheiden (siehe Kapitel \ref{sec:aboi} auf S.\pageref{sec:aboi}). Mit den Trainingsdaten wird dann eine \gls{kb} in Form eines \gls{mln}  trainiert, da diese mit ihren Eigenschaften (siehe Kapitel \ref{sec:mln} auf S.\pageref{sec:mln}) gut für die Wahrnehmung geeignet sind und über sie geschlussfolgert.  

\section{Gliederung der Arbeit}
\label{sec:gliederung}

\todo{WAS GEHT IN JEDEM KAPITEL SO AB?}

   

