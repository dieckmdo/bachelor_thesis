\graphicspath{{./images/}}      
\def\CHAPTERONE{./chapters/Chapter-1} 

\chapter{Einleitung}
\label{chap:introduction}
%	\input{\CHAPTERONE /motivation}
In den letzten Jahren haben autonome Roboter für Haushaltsumgebungen einen hohen Stellenwert in der Forschung eingenommen \todo{citation ?, sonst anders formulieren}. Ob es darum geht Essen zu kochen oder aufzuräumen, die möglichen Aufgaben sind nahezu unendlich und erleichtern so das Leben. Gerade in einer alternden Gesellschaft \todo{cite this?} ist Hilfe durch autonome Roboter eine erstrebenswerte Zukunft, auch wenn diese noch in weiter Ferne liegt. \par
Die für uns teils simpel erscheinenden Aufgaben, wie das einräumen einen Schranks, sind für einen Roboter schon komplexe Vorgänge. Er muss nicht nur die Objekte, die er einräumen soll, eindeutig identifizieren, sondern diese auch richtig und den äußerlichen Eigenschaften entsprechend greifen. So fässt man eine Packung Mehl anders an als eine Flasche Saft. All das muss der Roboter aus den Informationen, die seine Sensoren und Kameras erfassen, Schlussfolgern, um so die eigentliche Aufgabe, das Einräumen des Schranks, überhaupt beginnen zu können. \par
Umgebungen, die von Menschen bewohnt werden, sind jedoch starken nichtdeterministischen Schwankungen unterworfen. Es werden Objekte verrückt oder komplett aus dem Raum entfernt, sodass selbst wiederholt auszuführende Aufgaben niemals genau gleich ablaufen. Das erfordert von dem Roboter ein hohes Maß an Robustheit gegenüber Veränderungen seiner Umwelt. Um ihm dies zu ermöglichen, wird ihm eine Wissensbasis gegeben, in der er Wissen über und Beziehungen innerhalb seiner Umwelt abbilden kann. Zusammen mit diesem Wissen und den Informationen seiner Sensoren kann er nun Schlussfolgern (engl. Reasoning oder Inference), also Annahmen über seine Umgebung treffen. Das neue Objekt auf dem Küchentisch hat zu Beispiel die Form einer Flasche Saft, ist allerdings weiß. Es muss sich also um Milch handeln, die dann nicht in den Schrank, sondern den Kühlschrank kommt. \par 
Das Schlussfolgern ist also eine Zentrale Sache für den bestehen des Roboters in der veränderlichen Haushaltsumgebung. Das antrainieren solcher KBs ist jedoch Zeitaufwendig, da die Trainingsdaten vom Roboter in Trainingsläufen selber aufgenommen werden müssen. Da moderne Game Engines allerdings Szenen in Fotorealistischer Qualität in Echtzeit rendern können, bietet sich hier die Möglichkeit, die Umwelt des Arbeitsplatzes einen Roboters zu modellieren und dann hier diese Trainingsläufe laufen zu lassen. \par
Ziel dieser Arbeit ist es nun, zu testen, ob man mit Bildern aus einer Game Engine, eine KB antrainieren kann und das Schlussfolgern mit ihr gute Ergebnisse liefert. Das Institut für Artifical Intelligence (IAI) der Uni Bremen hat die Küche, in der mit einem PR2-Roboter geforscht wird, möglichst realistisch in die Unreal Engine (siehe Kapitel \ref{sec:unrealengine} auf S.\pageref{sec:unrealengine}) übertragen. Hier können nun Szenen erzeugt und dann in ein Robotersystem eingespeist werden. Das vom IAI entwickelte Framework ROBOSHERLOCK (siehe Kapitel \ref{sec:robosherlock} auf S.\pageref{sec:robosherlock}), wird dann Perzeptionsalgorithmen auf den Bildern laufen lassen und so Informationen über die gewünschten Objekte herausziehen. Mit diesen Informationen oder Wissen wird dann eine KB antrainiert. Diese ist in Form eines Markov Logic Networks (MLN) (siehe Kapitel \ref{sec:mln} auf S.\pageref{sec:mln}), da sich diese als besonders zuverlässig erwiesen haben \todo{Zitat}. Im letzten Schritt wird über die KB geschlussfolgert.    

\todo{etwas über Sharing knowledge of Robots siehe 3DNet-Intro}

\section{Gliederung der Arbeit}
\label{sec:gliederung}

\todo{WAS GEHT IN JEDEM KAPITEL SO AB?}


Roboter sollen immer schiwerigere Aufgaben erfüllen.

Um zu erfüllen -> Wahrnehmung der Umgebung zur Identifierung der nötigen Objekte.

Außerdem effiziente KB, die auch in unbekannte Umgebeung arbeiten kann. \par

In den vergangene Jahren gute Möglichkeiten zur Perzeption: ROBOSHERLOCK, um Objeckte verschiedener Eigenschaften zu verarbeiten.

Damit der Roboter damit arbeiten kann: KB trainieren.
Dieses Trainieren mit Daten oft lang und aufwendig.

Fotorealismus in Echtzeit Manipulation geboomt.

Statt echte Trainingsdaten nun "unechte" Daten aus Game Engine.

MLN wird als KB benutzt, weil ...

Untersuche ob die Perzeption an diesen Daten und das Reasoning über ein KB mit den Daten gute Ergebnisse liefert. Dann automatisierung von Testdaten ersteelung möglich. 
   

