\graphicspath{{./images/}}      
\def\CHAPTERONE{./chapters/Chapter-1} 

\chapter{Zielsetzung und Motivation}
\label{chap:motivation}
%	\input{\CHAPTERONE /motivation}

% Sinn und Zweck der Arbeit
% Was ist das Ziel? Was will ich herausfinden?
% Warum ist diese Arbeit sinnvoll?
% Was kann mit den Ergebnissen erreicht werden?

Perzeption und Reasoning von Robotersystem über "unechte" Bilder.

\section{Perzeption}

Wichitg für Umgebung einordung und objekterkennung. 

Algorithmen, Systeme, Trends \par

\subsection{Attribute Based Object Identification}

\cite{descObjbyAtr} \newline
Attribute von bekannten und unbekannten Objekten, um diese zu beschreiben, vergleichen und kategorisieren. Generalisierung von Kategorien einfacher mit Attributen. Attributklassifizierer nur mit features trainieren, die für das Attribut relevant sind (Farbe unwichitg für Form). Vermeidung von lernen des Korrelierenden Attributes. (wheel with Cars ->learn metallic. Wodden carriages??)\newline
Beschreibe unbekannte Objekte und lerne schnell Dinge über sie. \newline 
Semantische Attribute (Form, Material) nicht immer genug, also auch UnterscheidungsAttribute (Katzen und Hunde haben es, Pferde nicht).\newline
Mehrere Objekte teilen sich Attribute. Deshalb Verwendung von vorhergesehenen Attributen als features = kompakter, weniger features insgesamt. Damit nicht nur Erkennung sondern auch Unbekanntes beschreibbar. Fehlen oder Vorhandensein atypischer Attribute aufzeigbar. Modellernen mit wenig Beispielen und neue Kategorien auch aus Text allein ohne Bilder. \newline
Zeigt: \textit{selected feature} lernen (bei wheel, Bilder mit wheel und ohne, um lernen von metallic zu vermeiden) verbessert Klassifizierung (cross Category generalization). Mit allen feats scheint zu verwirren aka Traning set Korrelation bias. \newline
image features zu Kategorie bei Unbekannten hilflos. Also Attribute benutzen. So nicht nur Name sondern Eigenschaften benennbar.

\par

\cite{atrBasedObjIden} \newline
Identifizierung von Objekten basierend auf Aussehen und Namen Eigenschaften. \newline
Idnetifiziere Objekt in in Szene mit segmentierten Objekten basierend auf einem Satz, der Objekt beschreibt. \newline
110 Objekte in 12 Kategorien. \newline
Zeigt, dass Kombination verschiedener Attributklassifizierer die einzelnen Outperformed. Allerdings bei vielen Objekten nicht so gut. Man sollte örtliche oder relative(dunkler, kleiner) Beschreibungen zulassen. Bei Test mit minimalem subset der Attribute (nur Attribute um Objekte auseinander zu halten) zeigt hohe Robustheit von ABOI.
``We believe that the capability to learn from smaller sets
of examples will be particularly important in the context of
teaching robots about objects and attributes.'' \par


\cite{pronobis1} \newline
Zusammenführung von verschiedenen Informationsquellen mit verschiedenem Datentypen (Aussehen und Geometrie von Orten, Objekt informationen, Topologische Struktur, mensch Input). Kern: kompakte Repräsentation von komplexem, dynamischen, räumlichen Wissen. \newline
Kompakt: robuster mit Dynamik. \newline
Infer zusaätzliches Wissen aus Wissen + Sehen. \newline
KB in 4 Layers, categorical Layer hat Modelle von Objekten, Landmarks, room shapes. \newline
conceptual layer: Common-sense conceptual knowledge + Taxonomy von räumlichen Konzepten. Hier Info, dass kÜche enthält Milch und durch inferenz: sehe Milch also bin wahrscheinlich in Küche. Darstellung in realtionalen Beziehungen(kitchen has-object milk) und Instance-Wissen (obj1 is-a milk, place1 has-object obj1).\newline
Properties of Space: Ort bekommt Attribute, um ihn zu beschreiben.(objects, size, shape) \newline
Conceptual Map (KB): chain graph. Erlaubt reasoning über unbekannte Räume




\todo{PR2 Loóking at Things}

\subsection{Frameworks und andere Algorithmen?!?}

\cite{multimodalTemplate} \newline
Objekterkennung mit Template matching (Abgleich mit Vorlage, besser als featurePoint für gering texturierte Objs). Verbesserung von simplem TempMatch (Probleme mit viel Hintergrund Stördaten) durch multi-modalities(multi-Modalitäten) in Templates. Also zusätzliche Informationen mit Farbbild und Tiefenbild.\newline
Template matching kann schnell online gelernt werden, anders als andere Verfahren. \newline 
Farbbild Gradienten: robuster als Greyscale und bei untexturiert einzige Hinweise in Bildern. \newline
Tiefenbild: 3D Oberflächen Normale. \newline
Sehr Robust, hohe Erkkennrate: features/cues/Hinweise ergänzen sich gegenseitig.  \newline
Speed: template learning schnell: einfach multimodal features ectrahieren und teamplate anlegen. Schnell, da LINE-MOD nur abhängig von Nummer der features und nicht Größe der temapltes. \par


\cite{moped} \newline
framework zur Objekterkennung. Veruscht in Szenen mit hoher Komplexität zu arbeiten und mit geringer Latenz. \newline
Erreicht mit ICE(Iterative Clustering-Estimation): schätzt Featuregruppen (via SIFT, keypoints) ab, die wahrscheinlich zu Objekt gehören. SUcht Objekt-Hypothesen darin. Damit dann verbesserung der FeatGruppe und damit Hypo (``ICE iteratively executes clustering and pose estimation to progressively rene which features belong to each object instance, and to compute the object poses that best each object instance.'')... Einfache Paralellisierung möglich (Latenz bewältigen) \newline
Projection Clustering (Pose clustering algorithm): da Hypothesen von noisy Kamerabilder: Ausreißer! Jeder Typ wird ausgewertet und falsche Hypos mit den wahrscheinlich richtigen vereinigt. \newline
Ziel MOPED: Input:Bilder u DB. Output: Erkennung u Pose Estiamtion von Objekten. \newline
Learning der DB über 3D Modelle, die durch verschiedene Viewpoint Bilder der Objekte erstellt werden. \newline
Per image: 1 featExtraction, 1 matching, iter ICE ( 2 times), final cluster merging to merge not yet merged double detections. \newline
Medel creation braucht menschliche Interaktion. \par


\cite{3dnet} \newline
3DNet: DB mit 3D-CAD Modellen. + Framework: basiert auf PCL. bietet Deskriptoren, die mit 3D-Modellen traniiert werden und so echtzeit Objekt- und Objektklassen klassifizierung und Pose recognition bieten. In ROS integrierbar. Kinect Bilder zur Klassifikation. \newline
Objektklassen über Link zu WordNet. \newline 
Training mit 3D Modellen bietet Vorteile: 
\begin{itemize}
	\item Completeness
	\item Parameterizable
	\item sensor independence
	\item additional info available
\end{itemize}
Parameter und Deskriptoren Gewichtung Lernen ohne echte Szenen \newline
Intention von 3DNet: quelloffenes Framework mit TestDBs für Benchmarking zum objektiven Vergleich von Shape-Deskriptoren. \par


\citep{modelsWWW}
Lernen der Objekte von CAD-Modellen aus dem WWW. \newline
Detektion von Objekten durch Parts. (Lehne, Beine, ...) Besser bei Okklusion und verschiedenen Viewpoints. Mtching der Parts und Hough Voting, was das sein könnte. Dann welches CAD-Modell passt bestend? \newline
Lernen mit nur einem Modell pro Objekt, da andere ANsichten vom Roboter generiert werden können. Mit simuliertem Laserscan des Roboters über CAD-Modell um so Point CLoud zu kriegen.\newline
Repräsentation der Modelle: Vokabular der Parts + räumliche Anordnung in spezifischen Training Objekten. \newline
WWW hat gute CAD-Modelle von Büromöbeln (wg. Hobby, Hersteller). Experimente mit Büromöbeln. Sensor nimmt Point Cloud auf. \newline
Roboter in der Lage unbekannte Möbel zu erkennen und kategorisieren. \par


\section{Reasoning}

Wchitg für was soll cih tun? Was ist das? Kenne ich das schon.

Paar Algorithmen, Grundlagen, Trends (PRM) 

\section{Fotorealismus}

Vllt?

\section{Ziel der Arbeit}
\label{sec:goal}

Untersuche ob die Perzeption an diesen Daten und das Reasoning über ein KB mit den Daten gute Ergebnisse liefert. Dann automatisierung von Testdaten ersteelung möglich. 