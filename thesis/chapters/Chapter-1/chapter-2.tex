\graphicspath{{./images/}}      
\def\CHAPTERONE{./chapters/Chapter-1} 

\chapter{Zielsetzung und Motivation}
\label{chap:motivation}
%	\input{\CHAPTERONE /motivation}

% Sinn und Zweck der Arbeit
% Was ist das Ziel? Was will ich herausfinden?
% Warum ist diese Arbeit sinnvoll?
% Was kann mit den Ergebnissen erreicht werden?

Perzeption und Reasoning von Robotersystem über "unechte" Bilder.

\section{Perzeption}

Wichitg für Umgebung einordung und objekterkennung. 

Algorithmen, Systeme, Trends \par


\cite{descObjbyAtr}


\par

\cite{atrBasedObjIden} \newline
Identifizierung von Objekten basierend auf Aussehen und Namen Eigenschaften. \newline
Idnetifiziere Objekt in in Szene mit segmentierten Objekten basierend auf einem Satz, der Objekt beschreibt. \newline
110 Objekte in 12 Kategorien. \newline
Zeigt, dass Kombination verschiedener Attributklassifizierer die einzelnen Outperformed. Allerdings bei vielen Objekten nicht so gut. Man sollte örtliche oder relative(dunkler, kleiner) Beschreibungen zulassen. Bei Test mit minimalem subset der Attribute (nur Attribute um Objekte auseinander zu halten) zeigt hohe Robustheit von ABOI.
``We believe that the capability to learn from smaller sets
of examples will be particularly important in the context of
teaching robots about objects and attributes.'' \par


\section{Reasoning}

Wchitg für was soll cih tun? Was ist das? Kenne ich das schon.

Paar Algorithmen, Grundlagen, Trends (PRM) 

\section{Fotorealismus}

Vllt?

\section{Ziel der Arbeit}
\label{sec:goal}

Untersuche ob die Perzeption an diesen Daten und das Reasoning über ein KB mit den Daten gute Ergebnisse liefert. Dann automatisierung von Testdaten ersteelung möglich. 