\graphicspath{{./images/}}      
\def\CHAPTERONE{./chapters/Chapter-1} 

\chapter{Grundlagen}
\label{chap:software}
%	\input{\CHAPTERONE /motivation}

\section{Unreal Engine}
\label{sec:unrealengine}
Die von Epic Games\footnote{\url{https://www.epicgames.com}} entwickelte Unreal Engine\footnote{\url{https://www.unrealengine.com}} ist eine \gls{gameengine} zur Entwicklung von Videospielen und zur Erstellung von filmischen Erfahrungen \cite{featUnreal}. Mittlerweile befindet sich die Unreal Engine in ihrer vierten Generation und enthält nicht nur die \acrshort{engine} selbst, sondern auch eine Vielzahl an Werkzeugen, die alle in die \acrshort{engine} integriert oder angebunden sind, um den Workflow der Nutzer zu vereinfachen und effizienter zu gestalten. Erhältlich ist die \acrshort{engine} für die Betriebssysteme Windows, MacOS und Linux. \par

Im Kern bietet die Unreal Engine eine Möglichkeit zur Echtzeit Manipulation und Erstellung von Anwendungen wie Spielen. Dazu gehören zum Beispiel \gls{fotoreal}, VR Unterstützung und Entwicklung, ein Partikel- und VFX-System, sowie ein KI-System. Weitere Tools umfassen unter anderem einen Animator und einen Sequencer, der zur Erstellung und Editierung von filmischen Sequenzen benutzt werden kann. \par 

Seit Februar 2015 ist der C++ Quellcode der Unreal Engine offen verfügbar und die \acrshort{engine} selber für jeden kostenlos nutzbar, solange kein Gewinn in einer bestimmten Höhe erzielt wird \cite{freeUnreal}. Damit kann jeder an der Entwicklung der \acrshort{engine} teilhaben und sie für die persönlichen Zwecke individualisieren. Zusätzlich können auf dem Marketplace zusätzliche Inhalte und Plugins von Drittanbietern erworben und dann einfach in die\acrshort{engine} integriert werden, wodurch die Unreal Engine ein hohes Maß an Flexibilität und Individualisierung bietet. \par 

Für diese Arbeit interessant ist das \gls{fotoreal} in Echtzeit: die Unreal Engine bietet damit eine Möglichkeit echt wirkende Szenarien zu erstellen, aus denen dann Perzeptionsalgorithmen die Informationen herausfiltern, die für das Trainieren einer KB wichtig sind. Natürlich setzt das voraus, dass die Objekte in der zu rendernden Szene auch realistische Modelle besitzen. Dazu mehr in folgenden Abschnitten [Verweis].       

\subsection{Küchen Umgebung}
\label{subsec:kitchenenvironment}
Um in einer Game Engine ein zu spielendes \textit{Level} (oder \textit{Map}) mit Objekten, mit denen der Spieler interagieren kann, zu füllen, braucht man sogenannte \textit{Assets}. Dies sind 3D-Modelle, die in einem 3D-Modellierungsprogramm erstellt wurden oder 3D-Scans von echten Objekten, die dann in ein passendes Format überführt wurden. Auch wenn im Rahmen dieser Arbeit kein Bedarf besteht, das \textit{Level} zu spielen, werden trotzdem \textit{Assets} benötigt. Diese können dann so angeordnet werden, dass eine Kamera ein Bild von der Szene macht, über das dann die Perzeptionspipeline läuft. \par
Das IAI besitzt \textit{Assets} für eine realistische Küchenumgebung, die eine Nachbildung einer echten Küche im Institut ist. Sie wird im Rahmen des Projekts RobCoG\footnote{\url{http://www.robcog.org/}} ständig verbessert und erweitert. Das Ziel von RobCoG ist es Robotern über \textit{Serious Games} mit Commonsense und Wissen über Physik auszustatten. Da bei dem Erstellen und Abnehmen eines Bildes aus der \acrshort{engine} kein Bedarf besteht Schubladen zu öffnen, Objekte anzuheben oder die Herdplatten anzustellen, wird eine Version der Küche benutzt, die auf unnötige Funktionen verzichtet.\footnote{\url{https://github.com/robcog-iai/RobCoG/tree/dev-env}} \par
Die Objekte des täglichen Bedarfs, die im Rahmen dieser Arbeit erkannt werden sollen, sind auch Teil der Küchenumgebung. Es handelt sich bei ihnen um eingescannte echte Objekte, sodass sie einen hohen Detail- und Realitätsgrad besitzen und sich auch vom Roboter auch in der echten Küche finden lassen können.
 
\subsection{URoboVision}
\label{subsec:urobovision}
Damit die Bilder aus der Unreal Engine von der Perzeptionspipline in ROBOSHRLOCK verarbeitet werden können, wird ein Unreal Engine Plugin verwendet. URoboVision\footnote{\url{https://github.com/robcog-iai/URoboVision}} bietet eine Kamera, die die benötigten Informationen aus der Unreal Engine zieht und dann über eine TCP-Verbindung an ROBOSHERLOCK schickt. Dazu werden drei Bilder, ein Farbbild, ein Tiefenbild, und eine Objektmaske von der Szene erzeugt. Letzteres ist ein Bild, in dem jedes Objekt in einer anderen Farbe eingefärbt ist. Zusätzlich wird noch eine Map übermittelt, in der jede Farbe ein eindeutiger Objektname zugeordnet ist. Daraus kann später die Groundtruth für Objekthypothesen  ermittelt werden (mehr in \todo{wo ist das?}).   

\subsection{RSpawnBox}
\label{sec:rspawnbox}

Im Rahmen des Bachelor Projekts UnrealRobots im WiSe16/17 SoSe17 der Universität Bremen hat der Autor dieser Arbeit im VisionScanning-Subprojekt\footnote{\url{https://gitlab.informatik.uni-bremen.de/dieckmdo/P12-VisionScanning-UR16} GitLab Account der Universität Bremen notwendig} die RSpawnBox Klasse für die Unreal Engine implementiert. Diese bietet die Möglichkeit ausgewählte Objekte innerhalb einer Bounding Box an zufälligen Plätzen erscheinen zu lassen. Zusätzlich rotierte eine Kamera in anpassbarem Abstand und Winkel um die erschienenen Objekte. Im Rahmen dieser Arbeit wurde letzterer Teil der Klasse in stark modifizierter Form verwendet, um Bilder aus verschiedenen Ansichten der Szene mit der URoboVision Kamera aufzunehmen. Mehr dazu in \todo{wahrscheinlich Implementierung}.         

\section{ROS}
\label{sec:ros}
Das \gls{ros}\footnote{\url{http://www.ros.org}} ist ein quelloffenes, modular designtes Framework für die Entwicklung von Robotersoftware. \gls{ros} bietet dafür eine Kommunikationsebene über dem Betriebssystem der einzelnen Computer innerhalb eines Robotersystems und eine Menge hilfreicher Werkzeuge an \cite{ros}.\par 

Zu Grunde liegt dem eine Peer-To-Peer Topologie, sodass theoretisch jeder Prozess oder Computer mit jedem anderen kommunizieren kann. Dank eines Programmiersprachen-neutralen Designs ist dies auch über verschiedene Sprachen hinweg kein Problem und ermöglicht so die Berücksichtigung der Bevorzugung bestimmter Sprachen für bestimmte zu implementierende Probleme, wie auch Präferenzen des Entwicklers.\par

Die Kommunikation läuft über sogenannte \textit{Nodes}. Ein \textit{Node} ist ein Prozess der eine bestimmte Aufgabe bearbeitet. Der Datenaustauch findet nun über \textit{Topics} oder \textit{Services} statt. Beide haben eindeutig identifizierbare Namen und \textit{Nodes} können sich an ihnen als \textit{Publisher} anmelden oder als \textit{Subscriber} auf sie horchen. \textit{Nodes}, die als \textit{Publisher} tätig sind, veröffentlichen nun auf der gegebenem \textit{Topic} oder \textit{Service} Daten, während \textit{Subscriber} die Daten von den entsprechenden \textit{Topics} oder \textit{Services} erhalten. Der Unterschied besteht darin, dass \textit{Services} einmalig sind, während es mehrere \textit{Nodes} geben kann, die Daten auf einer bestimmten \textit{Topic} veröffentlichen. Der Austausch von Daten passiert dann in Form von \textit{Messages}. Dies sind feste Datenstrukturen, zum Beispiel die primitiven Datentypen integer und boolean, aber auch Arrays und \textit{Messages} selbst. \par

\gls{ros} kommt mit einer Reihe von Werkzeugen zur Entwicklung, die in jeweils eigenen Modulen implementiert sind, um so erhöhte Stabilität und verringerte Komplexität zu erreichen. Das Kernmodul, der \textit{ROS-master}, enthält somit nur die Kernfunktionalität. Die zusätzlichen Werkzeuge ermöglichen unter anderem das Debuggen einzelner \textit{Nodes}, die Visualisierung des Datenaustausches oder einzelner \textit{Topics}, das Starten ganzer \textit{Node}-Verbünde und die mehrfach Instanziierung von solchen, sowie das Erstellen von \textit{ROS-Packages}. Letzteres erlaubt das Aufteilen einzelner Funktionalitäten in Pakete und so das einfache Zusammenarbeiten mehrerer Entwickler. Jedes Paket kann jeweils seine eigenen Drittbibliothek-und Paketabhängigkeiten haben, sowie auch auch beliebig tief geschachtelt werden. Ein Paket kann also aus weiteren Paketen bestehen, wie zum Beispiel das im Rahmen dieser Arbeit verwendete ROBOSHERLOCK [Verweis]. Zum Zeitpunkt dieser Arbeit existieren über 3000 öffentliche \textit{ROS-Packages}.   

\section{ROBOSHERLOCK}
\label{sec:robosherlock}
ROBOSHERLOCK ist ein quelloffenes Framework zur Implementierung von Roboter Perzeptionssystemen. Dabei bietet ROBOSHERLOCK auch die Möglichkeit, Wissen zu akquirieren und darüber zu Schlussfolgern, als auch Fragen zu der wahrgenommenen Szene zu beantworten. Dabei bekommt der Roboter die Aufgabe nach Objekten einer bestimmten Beschreibung Ausschau zu halten und kann über diese dann zusätzliche Informationen erfahren. \par
ROBOSHERLOCK basiert auf dem \textit{unstructred information management}, was alle Daten als unstrukturiert ansieht, da ihnen Struktur und Semantik fehlt. Sogenannte \textit{Annotatoren} extrahieren nun Informationen aus den unstrukturierten Daten und geben ihnen so eine Bedeutung. Jeder \textit{Annotator} ist als \textit{Experte} auf eine bestimmte Information angesetzt, so kann ein der \textit{PlaneAnnotator} die den Objekten unterliegende Tischplatte finden oder der \todo{nettes Beisiel}. ROBOSHERLOCK bietet die Möglichkeit, dass mehrere \textit{Annotatoren} als Ensemble auf dem selben Daten arbeiten. So können sie sich gegenseitig ergänzen \todo{Verweis auf den einen Perceptionblock mit ensemble} und bessere Ergebnisse erzielen. Zusätzlich können durch die Kapselung bestimmter Aufgaben in \textit{Experten} neue Perzeptionsalgorithmen einfach als zusätzliche \textit{Annotatoren} eingepflegt werden. \par 
Als Eingabe erhält ROBOSHERLOCK verschiedene Sensordaten, die an sich unstrukturiert sind. Diese werden mit einer Struktur zum Speichern der \textit{Annotationen} der \textit{Annotatoren} und einem \textit{type system} in einer \textit{Common Analysis Structure (CAS)} gespeichert. ROBOSHERLOCK versucht nun Objekthypothesen zu erstellen, also welche Datenschnipsel könnten zu einem Objekt gehören. Diese Hypothesen werden in eindeutig identifizierbaren Strukturen, den \textit{Subjects of Anaylsis (SOFAs)}, gespeichert und zur \textit{CAS} hinzugefügt. Auf diesen können nun weitere \textit{Experten} laufen und so zusätzliche Informationen über die potenziellen Objekte zu erhalten. \par
\textit{Annotatoren} gibt es in zwei Varianten: \textit{general-purpose} und \textit{perception task-specific}. Erstere können auf jedem SOFA laufen, wie Farbe und Form, während letztere nur bei bestimmten Aufgaben zum Einsatz kommen. \newline
zwie AE formen: primitve and aggregate. \newline
1: hypthseses generation and SOFA creation \newline
2: annoate SOFAs (Annotators). Können auf andere Annoationen zugreifen und auf andere SOFAs. Dürfen inkonsistent und contradidtory sein -> auflösen durch folgendes reasoning und hypothesen testing u. ranking. Dazu Robo speichert: wer hat Annotation gemacht und wie sicher ist er. \newline
Mehrere AE zusammen mit flow controller formen Perzeptionspipeline. \par
Annoationen in Form von logischen Atomen -> kann Queries beantworten und reaoning über Annoattionen. \newline
ROBO hat engine zum lernen und schlussfolgern über Probailitische FP KBs, um inkonsistenz der Annotationen zu vernichten. \par 
meiste AEs mit PCL oder OpenCV 
  
\subsection{classifiers}
\label{sec:classifiers}


\section{MongoDB}
\label{sec:mongodb}


\section{Markov Logic Networks}
\label{sec:mln}

\subsection{pracmln}
\label{subsec:pracmln}
