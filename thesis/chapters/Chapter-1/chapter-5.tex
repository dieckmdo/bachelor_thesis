\graphicspath{{./images/}}      
\def\CHAPTERONE{./chapters/Chapter-1} 

\chapter{Umsetzung}
\label{chap:implementation}
%	\input{\CHAPTERONE /motivation}

In diesem Kapitel wird beschrieben, wie die einzelnen Arbeitsschritte umgesetzt wurden. Zuerst wird aufgezeigt, wie die Bildern in der \unreal erstellt wurden. Zur Vereinfachung werden diese  im Folgenden \textit{Unreal Bilder} genannt. Danach wird die verwendete \gls{ae} erläutert, die die Bilder annotiert. \todo{was kommt dann?}

\section{Erstellen der Unreal Bilder}
\label{sec:takingpics}
Um Unreal Bilder zum Trainieren eines \gls{mln} zur Verfügung zu haben, müssen diese in der \unreal erstellt werden. Dazu wurde die in Kapitel \todo{...} genannte Küchenumgebung benutzt. Hier wurden Objekte des täglichen Gebrauchs in der Küche manuell drapiert und mittels des \textit{URoboVision}-Plugins Bilder davon gemacht und an \robosherlock gesendet. Eine solche Anordnung von Objekten in der Küche wird \textit{Szene} genannt. Von jeder Szene wurden Bilder aus mehreren festgelegten Blickwinkeln mittels der \textit{SpawnBox} erstellt. \par 

\textbf{URoboVisionPlugin:} Das Plugin bietet einen in die Szene einfügbaren \todo{RCameraActor??}. Die verwendeten Parameter sind \todo{hier einfügren}. Damit wird an \robosherlock ein Farb- und Tiefenbild, sowie ein \textit{ObjectImage} und eine \textit{ObjectMap} gesendet. Auf dem ObjectImage ist jedes Objekt innerhalb der Szene in einer anderen Farbe eingefärbt. Die ObjectMap weist jeder im ObjectImage verwendeten Farbe den Namen des Objektes zu. So kann jedes sichtbare Objekt eindeutig identifiziert werden.   \par 

\textbf{Aufnahme aus verschiedene Blickwinkeln:} Die veränderte \textit{SpawnBox} besitzt eine \textit{BoxVolume} aus der vorherigen Iteration. Sie wurde beibehalten, da sie eine visuelle Markierung bietet, wo die Objekte drapiert werden sollten, damit sie im späteren Bild noch zu sehen sind und nicht außerhalb den Blickfeldes der Kamera liegen. Der Mittelpunkt der Box dient als Rotationspunkt, um den eine referenzierte Kamera basierend auf einigen Parametern rotiert wird. Die Kamera wird dabei nach einer \todo{bestimmten Zeit} in einem einstellbaren Abstand $x$-mal um einen einen einstellbaren Winkel von ihrem aktuellen Standort um den Mittelpunkt rotiert. Der Startpunkt der Kamera wird über den Mittelpunkt der Box ermittelt und befindet sich \textit{vor} der Box, also in \todo{x richtung und y=0}, sodass die Kamera nur Bilder abhängig von den Parametern der \textit{SpawnBox} macht, und nicht irgendwo anders im Raum. Die Höhe der Kamera kann ebenfalls eingestellt werden, wobei der Mittelpunkt der Box als ebenerdig, also die Höhe 0, angesehen wird. Die Kamera kann auch auf zwei unterschiedlichen Höhen Operieren, wobei der Winkel und die Rotationsschritte unabhängig voneinander eingestellt werden können. \todo{Das folgende ist nicht optional sondern default. Das "normale" ist nicht mehr vorhanden.}Um die Bedienung mit der Box als Markierung zu erleichtern, wurde es außerdem möglich gemacht, die Kamera nicht vor der Box starten zu lassen, sondern diesen Ort als Mittelpunkt der Rotation anzusehen. Werden zum Beispiel drei Rotationsschritte mit einem Winkel von $30^\circ$ erwünscht, wären die Standorte der Kamera bei der normalen Einstellung \todo{vor und jeweils $30^\circ$ weiter links  (abhängig von richtung siehe oben?!?)} von der Box. Bei der anderen Einstellung würde die Kamera $30^\circ$ rechts vom normalen Startpunkt beginnen, dann auf den eigentlichen Startpunkt kommen und dann $30^\circ$ nach links schwenken. So wird es einfacher Blickwinkel zu generieren, die dem Fahren und Umsehen eines Roboters ähnlicher sind. \todo{Cam blickt immer auf Mittelpunkt!!!} \newline
Die einzelnen Parameter und ihre Funktion werden in Tabelle \ref{tab:spawnboxParams} vorgestellt.

\begin{table}
\begin{tabularx}{\textwidth}{XXXX}
\textbf{Datentyp} & \textbf{Parameter}  & \textbf{Beschreibung} & \textbf{Wert} \\ \hline
UBoxComponent     & SpawnVolume         & Eine Box, die hilft die Objekte zu platzieren. Der Mittelpunkt ist der Rotationspunkt der Kamera. & -- \\ 
ARGBDCamera       & ScanCamera          & Die Kamera aus dem URoboVision Plugin & Wird manuell gesetzt. & -- \\
int               & CameraRadius        & Der Radius des Kreises auf dem die Kamera sich um den Mittelpunkt des SpawnVolume bewegt in UnrealEinheiten. & 70 \\
int               & CameraHeight        & Die Höhe der Kamera. Der Wert wird auf die Höhe des Mittelpunktes der SpawnVolume addiert in UnrealEinheiten.  & 35 \\
float             & Angle               & Der Winkel, um den die Kamera pro Rotationsschritt rotiert wird in Grad. & 45 \\
float             & UpdateTime          & Die Zeit die vergeht bis die Kameraposition aktualisiert wird in Sekunden. & 3 \\
int               & NoOfViewpoints      & Die Anzahl der Blickwinkel, also Rotationsschritte. & 3 \\
float             & HeightAngle         & Soll eine zweite Höhe benutzt werden, beschreibt dies den Winkel für diese zweite Höhe. & 45 \\
int               & HeightOffset        & Der Unterschied der zweiten Höhe zur Ersten. & 35 \\
int               & NoOfHeightViewpoints & Die Anzahl der Rotationsschritte für die zweite Höhe. & 2 \\
bool              & bUseHeigthViewpoints & Ob die zweite Höhe verwendet werden soll. & true \\
bool              & bUseMirror          & Wenn wahr, wird der eigentliche Startpunkt der Kamera als Spiegelpunkt für die Rotation benutzt.  & \\  
\end{tabularx}
\caption{Parameter der \textit{RSpawnbox}}
\label{tab:spawnboxParams}
\end{table}

\todo{den Algorithmus der SpawnBox}
\begin{algorithm}[H]
\begin{algorithmic}
\FOR{$i=0$ to $10$}
\STATE carry out some processing
\ENDFOR
\end{algorithmic}
\caption{Blickwinkelberechnung der RSpawnBox}
\end{algorithm}


\begin{algorithm}[H]
\begin{lstlisting}
void ASpawnBox::UpdateCamera()
{
	if (ScanCamera)
	{
    if( ProcessedViewpoints == 0)
    {
      GetWorld()->GetTimerManager().ClearTimer(UpdateTimerHandle);
      bProcessHeightViewpoints = false;
      return;
    }


    float i;
    if (ProcessedViewpoints == NoOfViewpoints + NoOfHeightViewpoints)
    {
      i = NoOfViewpoints / 2;
      AngleAxis -= (HeightAngle * ((NoOfViewpoints % 2 != 0) ? i : (0.5 * i)));
    }
    else if (ProcessedViewpoints == NoOfHeightViewpoints)
    {
      AngleAxis = 0.0f;

      i = NoOfHeightViewpoints / 2;
      AngleAxis -= (HeightAngle * ((NoOfHeightViewpoints % 2 != 0) ? i : (0.5 * i)));
      bProcessHeightViewpoints = true;
    }
    else
    {
      // normal step
      // compute new location on a circle around the SpawnVolume
      AngleAxis += bProcessHeightViewpoints ? HeightAngle : Angle;
      if( AngleAxis >= 360.0f) { AngleAxis = 0; }
    }

    --ProcessedViewpoints;
    FVector NewLocation = SpawnVolume->GetComponentLocation();

    FVector RotateValue = FVector(CameraRadius, 0, 0).RotateAngleAxis(AngleAxis, FVector::UpVector);


    NewLocation.X += RotateValue.X;
    NewLocation.Y += RotateValue.Y;
    NewLocation.Z += bProcessHeightViewpoints ? (CameraHeight + HeightOffset) : CameraHeight;

    // the camera looks to the midpoint of the SpawnVolume
    FRotator Rot = UKismetMathLibrary::FindLookAtRotation(NewLocation, SpawnVolume->GetComponentLocation());

    ScanCamera->SetActorLocationAndRotation(NewLocation, Rot);
	}
}
\end{lstlisting}
 \caption{SpawnBox}
\end{algorithm}

 \par

Um die Unreal Bilder zu erstellen, wurde die Parameter so eingestellt, dass die Kamera auf einer Höhe von ... und ... um die Szene rotiert. Es werden dabei 5 Bilder gemacht, 3 aus Höhe1 zwei aus Höhe2. etc.    Abspeicherung in MongoDB.

\section{Analysis Engine}
\label{sec:analysisengine}
Die Unreal Bilder können  von einer Perzeptionspipeline in \robosherlock annotiert werden und so die Objektattribute als logische Prädikate zum Training eines \gls{mln} erhalten werden. Dazu werden die Unreal Bilder aus der Datenbank herausgelesen und als Eingabe benutzt. Auf diesen werden zuerst eine Reihe von \textit{hyptheses generators} zur Segmentierung laufen lassen mit denen die Objekthypothesen erstellt werden. Dabei werden jedoch auch Objekthypothesen zu Formen und Teilen des Bildes erstellt, die keine Objekte sind, wie die Front eines Schrankes, oder keine für die Arbeit relevanten Objekte, wie die Herdplatte. Der \texttt{UnrealGTAnnotator} filtert diese \textit{Cluster} nun heraus und annotiert zu den übrig gebliebenen die \gls{gt}. Die gebliebenen Hypothesen werden von den Annotatoren auf Eigenschaften analysiert und die Ergebnisse als Annotationen in der \gls{cas} gespeichert. In einem letzten Schritt werden die Annotationen von dem \texttt{MLNInferencer} als logische Prädikate in einer Textdatei ausgeben. \par 

Im Folgenden wird der \texttt{UnrealGTAnnotator} und seine Implementierung erläutert. Danach werden die Annotatoren der Perzeptionspipeline vorgestellt. Tabelle \ref{tab:annotators} bietet eine kurze Übersicht über die Annotatoren, die resultierenden logischen Prädikate und die möglichen Werte.

\subsection{UnrealGTAnnotator}
Der \texttt{UnrealGTAnnotator} filtert alle falschen Objekthypothesen heraus und annotiert für die übrig gebliebenen die \gls{gt}. Dazu wird das \textit{ObjectImage} der Unreal Bilder verwendet. Der Algorithmus wird im folgenden beschrieben und ist in Algorithmus \ref{alg:UnrealGTAnnotator} in Pseudocode dargestellt. Die einzelnen Teilbilder der Unreal Bilder sind deckungsgleich, das heißt die Koordinaten eines Pixels beschreiben in jedem Teilbild den gleichen Punkt. Für jeden Pixel eines Clusters wird nun seine Farbe im ObjectImage betrachtet und so die Anzahl der Pixel, die sich eine Farbe Teilen gezählt. Nun wird für die am meisten vorkommende Farbe in der ObjectMap nachgeschlagen, zu welchem Objekt die Farbe gehört. Der Name des Objektes wird mit einer anderen \todo{Map/Array} verglichen, die die Namen der verwendeten Objekte hält (siehe eine Liste der Namen \todo{hier}). Wird eine Übereinstimmung gefunden, wird für den Cluster eine \gls{gt}-Annotation zur \gls{cas} hinzugefügt. Die \gls{gt}, die als Annotation hinzugefügt wird, ist \todo{InstanzName oder Klasse. ...} Wird keine Übereinstimmung gefunden, wird der Cluster verworfen. Es kann allerdings vorkommen, dass eine Objekthypothese korrekt ist, das entsprechende Objekt in seinem ObjectImage jedoch nicht die meisten Pixel belegt. Dies ist vor allem bei Besteck, wie Gabeln und Messern zu beobachten. Um dem vorzubeugen, wird rekursiv auch die zweit und dritt häufigste Farbe auf eine Übereinstimmung geprüft. Eine Rekursion von drei, hat sich als ausreichend erwiesen, da das Besteck meistens auf einem Tisch liegt, der Cluster also nur aus zwei Farben besteht, und damit schon bei dem zweiten Rekursionsschritt gefunden wird. Auch wenn ein weiteres größeres Objekt mit im Cluster zu finden ist, wird das Besteck spätestens im dritten Rekursionsschritt gefunden. Ein vierter Durchlauf würde den Algorithmus nur unnötig verlangsamen, da später verworfene Cluster dann immer die gesamte Rekursion komplett durchlaufen würden. 

\todo{den Algorithmus des UnrealGTAnnotator}
\begin{algorithm}[H]
\begin{algorithmic}
\FORALL{clusters $c$ from the scene}
\STATE carry out some processing
\ENDFOR
\end{algorithmic}
\caption{UnrealGTAnnotator}
\label{alg:UnrealGTAnnotator}
\end{algorithm}


\begin{algorithm}[H]
\begin{lstlisting}
 TyErrorId processWithLock(CAS &tcas, ResultSpecification const &res_spec)
  {
    MEASURE_TIME;
    outInfo("process begins");
    rs::SceneCas cas(tcas);
    rs::Scene scene = cas.getScene();

    cas.get(VIEW_COLOR_IMAGE_HD, color);
    cas.get(VIEW_OBJECT_IMAGE_HD, objects);
    cas.get(VIEW_OBJECT_MAP, objectMap);

    disp = color.clone();

    std::vector<rs::Identifiable> cleanedClusters;

    std::vector<rs::Cluster> clusters;
    scene.identifiables.filter(clusters);

    outInfo("Found " << clusters.size() << " clusters");

    for(auto cluster : clusters)
    {
      cv::Rect roi;
      rs::conversion::from(cluster.rois.get().roi_hires.get(), roi);

      cv::Mat imageRoi = cv::Mat(objects, roi);

      // cv::Vec3b not a possible key, so using the name of the object associated with the color.
      std::map<std::string, int> colorCount;

      for(int r = 0; r < imageRoi.rows; ++r)
      {
        const cv::Vec3b *imageIt = imageRoi.ptr<cv::Vec3b>(r);

        for(int c = 0; c < imageRoi.cols; ++c, ++imageIt)
        {
          std::map<std::string, cv::Vec3b>::iterator objMapIt;

          // find entry in objectMap matching color
          for(objMapIt = objectMap.begin(); objMapIt != objectMap.end(); ++objMapIt)
          {
            if(objMapIt->second == *imageIt)
            {
              // add new object occurence to colorCount or increase value
              std::pair<std::map<std::string, int>::iterator, bool> countIt = colorCount.insert(std::pair<std::string, int>(objMapIt->first, 1));
              if(countIt.second == false)
              {
                ++countIt.first->second;
              }
            }
          }
        }
      }

      // for small objects, that do not have the most pixels in their roi (cutlery) search for a suitable object in a depth of 3.
      for(int i = 0; i < 3; ++i)
      {
        std::string foundObj = "";
        if(colorCount.empty())
        {
          break;
        }

        std::string mostColor = getObjectWithMostOccurences(colorCount);
        for(auto object : posObjects)
        {
          if(!mostColor.empty() && mostColor.find(object) != std::string::npos)
          {
            foundObj = object;
            break;
          }
          else
          {
            colorCount.erase(mostColor);
          }
        }

        if(foundObj != "" && !foundObj.empty())
        {
          rs::GroundTruth gt = rs::create<rs::GroundTruth>(tcas);
          rs::Classification classification = rs::create<rs::Classification>(tcas);
          classification.classification_type.set("ground_truth");
          classification.classname.set(foundObj);
          classification.classifier.set("UnrealEngine");
          classification.source.set("UnrealGTAnnotator");
          gt.classificationGT.set(classification);
          cluster.annotations.append(gt);

          cleanedClusters.push_back(cluster);

          drawResults(roi, foundObj);
          break;
        }
      }
    }

    outInfo("Reduced to " << cleanedClusters.size() << " clusters");
    scene.identifiables.set(cleanedClusters);
    return UIMA_ERR_NONE;
  }

  // find the most occuring color
  std::string getObjectWithMostOccurences(std::map<std::string, int> map)
  {
    std::map<std::string, int>::iterator mostColor = map.begin();
    for(std::map<std::string, int>::iterator it = map.begin(); it != map.end(); ++it)
    {
      if(it->second > mostColor->second)
      {
        mostColor = it;
      }
    }
    return mostColor->first;
  }
\end{lstlisting}
 \caption{UnrealGTAnno}
\end{algorithm}


\subsection{Annotatoren}

Im Folgenden werden die Annotatoren vorgestellt. Sie analysieren die Cluster der Objekthypothesen auf Eingenschaften. \todo{Warum shape zweimal?} Sie sind in einer Perzeptionspipeline zusammengefasst und kommen auch in der Reihenfolge zum Einsatz, in der sie hier vorgestellt werden. Tabelle \ref{tab:annotators} bietet eine kurze Übersicht.

\begin{table}
\begin{tabularx}{\textwidth}{lXX}
\textbf{Annotator}				& \textbf{MLN-Prädikat}	& \textbf{Werte}	\\ \hline 
Cluster3DGeometryAnnotator		& size(cluster, size)	& small, medium, large	\\ \hline 
PrimitiveShapeAnnotator			& shape(cluster, shape)	& box, round, flat 	\\ \hline 
ClusterColorHistogramCalculator  & color(cluster, color)	& red, yellow, green, cyan, blue, magenta, white, black, grey \\ \hline 
GogglesAnnotator				& goggles\_Product(cluster, logo) \newline googles\_Text(cluster, text) 	& 	variabel	\\ \hline 
CaffeAnnotator					& --					& --	\\ \hline 
PCLFeatureExtractor				& --					& --	\\ \hline 
RfAnnotator						& shape(cluster, shape)	&	box, cylindrical, disk, flat, sphere, other	\\ \hline 
SVMAnnotator					& instance(cluster, i)	&		\\ \hline 
\end{tabularx}
\caption[Kurzübersicht der Annotatoren]{Eine Kurzübersicht über die Annotatoren und die korrespondieren logischen Prädikate für das \gls{mln}.}
\label{tab:annotators}
\end{table}

\subsubsection{Cluster3DGeometryAnnotator}
Basierend auf der Distanz zwischen Extrempunkten der Objekte, die mit der Entfernung zur Kamera normalisiert wurde, wird dem Objekt eine Größe zugeordnet. Es wird zwischen $big$ und $small$ unterschieden. Das logische Prädikat ist von der Form $size(cluster,  size)$. 

\subsubsection{PrimitiveShapeAnnotator} 
Der \texttt{PrimitiveShapeAnnotator} berechnet die primitive geometrische Form der Objekte.  Dabei wird zwischen $round$ und $box$ unterschieden. Das logische Prädikat ist von der Form $shape(cluster,  shape)$.
   
\subsubsection{ClusterColorHistogramCalculator}
Der \texttt{ClusterColorHistogramCalculator} berechnet für jeden Cluster die Farbverteilung. Daraus bildet er ein Farbhistogramm, das der \gls{cas} hinzugefügt wird. Die unterschiedenen Farben sind: \todo{Farben}. Die am häufigsten auftretende Farbe wird später als logisches Prädikat in der Form $color(cluster,  color)$ ausgegeben. Je nach Häufigkeitsverteilung kann ein Cluster auch zwei Farben haben, die in dem Fall auch beide ausgegeben werden. 

\subsubsection{GogglesAnnotator}
Der \texttt{GooglesAnnotator} benutzt den Webservice \textsc{Google Googles}. Dazu werden die Bilder des Clusters an den Server des Webservices gesendet, der nun Webseiten mit ähnlichen Bildern sucht und zurückgibt. Aus den Ergebnissen werden Logo, Text und Textur Annotationen erstellt. Die Werte dafür sind abhängig von den Ergebnissen. Da es häufig vorkommt, dass keine Ergebnisse vorliegen, werden in solchen Fällen auch keine Annotationen zur \gls{cas} hinzugefügt. Die logischen Prädikate sind $goggles\_Product(cluster, logo)$, $googles\_Text(cluster, text)$ und $googles\_texture(cluster, t)$. 

\subsubsection{CaffeAnnotator}
Für den \texttt{CaffeAnnotator} wird das Deep-Learning Framework \textsc{Caffe}\footnote{\url{http://caffe.berkeleyvision.org/}} benötigt. Der Annotator extrahiert Merkmale aus den Bildern und speichert diese als Annotation in der \gls{cas}. Sie werden später für die Klassifikation benötigt.  

\subsubsection{PCLFeatureExtractor}
Wie der \texttt{CaffeAnnotator} fügt der \texttt{PCLFeatureExtractor} Merkmale als Annotationen zur \gls{cas} hinzu. Er extrahiert \gls{vfh}-Merkmale mittels der \gls{pcl} aus den Bildern. Diese werden für den Formklassifizierer benötigt.

\subsubsection{RfAnnotator}
\todo{check} Der \texttt{RFAnnotator} klassifiziert die Form der Objekte. Der verwendete \gls{rf} wurde für die Masterarbeit von Rakibul Islam \cite{rakib} mit einem Objektset der Washington University trainiert. Der Klassifizierer unterscheidet dabei folgende Formen: \todo{Formen}. Die logischen Prädikate sind von der Form  $shape(cluster, shape)$. Mehr Informationen zu Klassifizierern sind in Kapitel \ref{sec:classifiers} auf S. \pageref{sec:classifiers} zu finden.

\subsubsection{SVMAnnotator}
\todo{check} Der \texttt{SVMAnnotator} klassifiziert die Instanz der Objekte. Dazu wurde eine \gls{svm} mit echten Bildern \todo{zu finden hier} der in den Experimenten vorkommenden Objekten trainiert. \todo{Für eine Liste der möglichen Instnazen siehe hier...}. Die logischen Prädikate sind von der Form  $instance(cluster, i)$. Mehr Informationen zu Klassifizierern sind in Kapitel \ref{sec:classifiers} auf S. \pageref{sec:classifiers} zu finden.

\subsubsection{MLNInferencer}
Der \texttt{MLNInferencer} liest die Annotationen aus der \gls{cas} aus gibt sie als logische Prädikate in einer Textdatei aus. Diese Dateien werden Datenbank genannt und bilden die Trainingsdaten für ein \gls{mln}. Der \texttt{MLNInferencer} kann die Prädikate sowohl in \gls{fo} ausgeben, als auch in \textit{Fuzzy-Logik}. Dabei werden für jedes Prädikat die Konfidenzen der Annotatoren mit ausgegeben.

\begin{algorithm}[H]
\begin{lstlisting}
if(isCodeIsFancy){
// blah
}
\end{lstlisting}
 \caption{How to write algorithms}
\end{algorithm}
