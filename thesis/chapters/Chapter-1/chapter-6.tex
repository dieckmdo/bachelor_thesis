\graphicspath{{./images/}}      
\def\CHAPTERONE{./chapters/Chapter-1} 

\chapter{Experimente}
\label{chap:experiments}
%	\input{\CHAPTERONE /motivation}

Im folgenden Kapitel werden die Experimente beschrieben, die im Rahmen dieser Arbeit durchgeführt wurden, um die Objekterkennung in fotorealistischen Bildern zu evaluieren. \todo{ausweiten?} 

\section{Unreal-Bilder}
\label{onlyUnrealImages}
In einem ersten Experiment werden nur Unreal-Bilder zum trainieren und testen des \gls{mln} benutzt. Die logischen Prädikate wurden dazu mit der in Kapitel \ref{sec:analysisengine} vorgestellten \gls{ae} aus den Unreal-Bildern extrahiert. \todo{scene Prädikat beschreiben} Das Folgende \gls{mln} beschreibt die Zusammenhänge der einzelnen Informationen der Annotatoren und der Objektklassen: \todo{mln einfügen}

Um dieses Modell zu evaluieren wird 10-fache Kreuzvalidierung durchgeführt. Dies verhindert Überanpassung des Modells, also die Anpassung an die Trainingsdaten und damit einen Verlust der Generalität des Modells.    
  
Leider sind LinuxCup und YellowPlate nicht mehr zur Hand. Plate mit weißer Platte ersetzt. Cup fehlt.

\begin{table}
\rowcolors{1}{}{lightgray}
\begin{tabularx}{\textwidth}{llccccc}
\textbf{Instanz}  				& \textbf{Klasse}	& \textbf{breakfast}	& \textbf{fridge}	& \textbf{cooking}	& \textbf{Unreal} & \textbf{Real} \\ \hline
AlbiHimbeerJuice				& Juice				& +			& +			& +			& 14	& 18	\\
BlueCeramicIkeaMug				& DrinkingMug		& +			& 			&			& 8		& 9\\
BlueMetalPlateWhiteSpeckles		& DinnerPlate		& 			& 			&	+		& 9		& 10\\
BluePlasticBowl					& Bowl				& +			& 			&			& 9		& 9\\
BluePlasticFork					& Fork				& 			& 			&	+		& 8		& 9\\
BluePlasticKnife				& Knife				& 			& 			&	+		& 8		& 9\\
BluePlasticSpoon				& Spoon				& +			& 			&			& 8		& 11\\
CupEcoOrange					& Cup				& +			& 			&	+		& 12	& 15\\
EdekaRedBowl					& Bowl				& +			& 			&			& 8		& 9\\
ElBrygCoffee					& Coffee			& +			& 			&			& 8		& 9\\
JaMilch							& Milk				& +			& +			&			& 13	& 14\\
JodSalz							& TableSalt			& 			& 			&	+		& 9		& 9\\
KelloggsCornFlakes				& BreakfastCereal	& +			& 			&			& 8		& 9\\
KelloggsToppasMini				& BreakfastCereal	& +			& 			&			& 8		& 9\\
KnusperSchokoKeks				& BreakfastCereal	& +			& 			&			& 8		& 8\\
KoellnMuesliKnusperHonigNuss	& BreakfastCereal	& +			& 			&			& 8		& 8	\\
LargeGreySpoon					& Spoon				& 			& 			&	+		& 9		& 8	\\
LinuxCup						& DrinkingMugin		& +			& 			&			& 8		& 0\\
LionCerealBox					& BreakfastCereal	& +			& 			&			& 8		& 9	\\
MarkenSalz						& TableSalt			& 			& 			&	+		& 9		& 8	\\
MeerSalz						& TableSalt			& 			& 			&	+		& 9		& 8	\\
MondaminPancakeMix				& PancakeMix		& 			& 			&	+		& 9		& 8	\\
NesquikCereal					& BreakfastCereal	& +			& 			&			& 8		& 9	\\
PfannerGruneIcetea				& Tea-Iced			& 			& +			&	+		& 13	& 14\\
PfannerPfirsichIcetea			& Tea-Iced			& 			& +			&	+		& 14	& 13\\
RedMetalBowlWhiteSpeckles		& Bowl				& +			& 			&			& 9		& 8\\
RedMetalCupWhiteSpeckles		& Cup				& +			& 			&			& 8		& 10\\
RedMetalPlateWhiteSpeckles		& DinnerPlate		& 			& 			&	+		& 9		& 9\\
RedPlasticFork					& Fork				& 			& 			&	+		& 8		& 9\\
RedPlasticKnife					& Knife				& 			& 			&	+		& 8		& 9\\
RedPlasticSpoon					& Spoon				& +			& 			&			& 9		& 10\\
ReineButterMilch				& Buttermilk		& +			& +			&			& 13	& 14\\
SeverinPancakeMaker				& PancakeMaker		& 			& 			&	+		& 10	& 10\\
SiggBottle						& DrinkingBottle	& 			& 			&	+		& 9		& 9\\
SlottedSpatula					& Spatula			& 			& 			&	+		& 9		& 8\\
SojaMilch						& Milk				& +			& +			&			& 13	& 14\\
SpitzenReis						& Rice				& 			& 			&	+		& 9		& 9\\
TomatoAlGustoBasilikum			& TomatoSauce		& 			& 			&	+		& 8		& 8\\
TomatoSauceOroDiParma			& TomatoSauce		& 			& 			&	+		& 9		& 8\\
VollMilch						& Milk				& +			& +			&			& 13	& 14\\
WeideMilchSmall					& Milk				& +			& +			&			& 13	& 14\\
WhiteCeramicIkeaBowl			& Bowl				& +			& 			&			& 8		& 9\\
YellowCeramicPlate				& DinnerPlate 	    & 			& 			&	+		& 9		& 6\\ \hline
\textbf{Insgesamt: 43}				& \textbf{20}		& \textbf{23} & \textbf{8} & \textbf{22} & & \\
\end{tabularx}
\caption[Objekte und ihre Verteilung in den Szenen]{Die Tabelle enthält alle Objekte und welchen generelleren Klassen sie zugeordnet sind, sowie die Zuordnung zu den Szenarien und wie häufig sie insgesamt in den Szenen vorkommen.}
\label{tab:objects}
\end{table}