\graphicspath{{./images/}}      
\def\CHAPTERONE{./chapters/Chapter-1} 

\chapter{Fazit}
\label{chap:fazit}
%	\input{\CHAPTERONE /motivation}

In dieser Bachelorarbeit wurde untersucht, ob sich fotorealistische, synthetisch erzeugte Bilder für die Klassifikation von Objekten eignen. Dazu wurde zu Beginn eine Einführung in die verschiedenen für die Arbeit relevanten Themen gegeben. Nachdem verschiedene Perzeptionsalgorithmen und \glspl{framework} zur Objekterkennung und Klassifikation vorgestellt wurden, wurde die Attribut-basierte Objekterkennung erläutert. Sie bildet die Grundlage der Klassifikation in dieser Arbeit. Danach wurde das Thema Virtuelle Realität und ihr Nutzen für die Wahrnehmung von Robotern vorgestellt. Ein weiterer Abschnitt befasste sich mit dem Thema Inferenz bei der Perzeption von Robotern. Im Anschluss wurden die benutzten Programme und Konstrukte beschrieben. Mit der \unreal wurden die synthetischen Bilder in einer Virtuellen Realität erzeugt. \robosherlock wurde zur Interpretation der Bilder verwendet. \gls{mln} repräsentierten eine probabilistische \gls{kb}, aus der mittels Inferenz die Objektklassifikation geschlussfolgert werden konnten. In Kapitel \ref{chap:implementation} wurden die im Rahmen der Arbeit implementierten Lösungen erläutert, synthetische Bilder zu erzeugen und aus ihnen die \gls{gt} der Objekte zu ziehen. Diese wurde zusammen mit den anderen Annotationen der vorgestellten \gls{ae} als logische Prädikate ausgegeben. Mit ihnen wurden in verschiedenen Experimenten \glspl{mln} trainiert und getestet. \par

Die Ergebnisse der verschiedenen Experimente zeigen, dass sich synthetische Bilder durchaus für die Objektklassifikation eignen. So wurde bei den realen Bilder eine Klassifikationsrate erreicht, die ähnlich der Ergebnisse aus \cite{pr2looking} ist. Darin wurde ein \gls{mln} mit echten Bildern trainiert und getestet. Es ist also durchaus möglich, synthetische Bilder als Trainingsdaten zu verwenden, um so dem hohen Aufwand der Erstellung von realen Bildern vorzubeugen. Die Experimente zeigen aber auch, dass man nicht auf reale Bilder verzichten sollte. Denn schon ein Trainingsset mit synthetischen Bildern und realen Bildern im Verhältnis 5:1 verbessert die Rate enorm. Es lässt sich außerdem festhalten, dass die Klassifikation nach Klassen bessere Ergebnisse liefert als nach Objektinstanzen. Dies ist vor allem dadurch der Fall, dass Objekte innerhalb einer Klasse ähnliche visuelle Eigenschaften besitzen und dadurch häufiger verwechselt werden. In einem solchen Fall bleibt die Einordnung zur Klasse allerdings korrekt und ist nur ein Problem bei der Instanzenklassifikation.   \newline
Gleichzeitig unterstützen die Ergebnisse aber auch die Idee der besseren Klassifikation durch die Zusammenführung von Informationen. Die einzelnen Experten, repräsentiert durch die \robosherlock Annotatoren, zeigen deutliche Stärken und Schwächen. Vereint man sie jedoch in einem \gls{mln} unterstützen sie sich gegenseitig. Zu sehen ist dies unter an den deutlich höheren Klassifikationsraten der \gls{mln} Experimente gegenüber des Ergebnissen der \glspl{klassifikator}. Aber auch in einem \gls{mln} kann dies nachgewiesen werden. 

\section{Ausblick}
\label{chap:ausblick}   

Da synthetische Bilder sich zur Klassifikation eignen, ist der nächste Schritt, die Erstellung der Bilder stärker zu automatisieren. Da die einzelnen Szenen in dieser Arbeit noch manuell erstellt wurden, sollte es durch eine Automatisierung des Prozesses möglich sein, mehr Bilder in kürzerer Zeit zu erstellen. Eine Möglichkeit dies zu erreichen, wäre die Wiedereinführung der ursprünglichen Funktionalität der \textit{RSpawnBox}. Diese ließ ausgewählte Objekte an zufälligen Orten innerhalb ihrer Grenzen erscheinen. Ein anderes Vorgehen wäre es, echte Bilder aufzunehmen und aus ihnen synthetische Variationen zu erzeugen. Dazu müssten die Positionen und Rotationen der Objekte aus der echten Bildern extrahiert werden, und dann in das Koordinatensystem der \unreal überführt werden. An diesen Punkten könnten nun bestimmte Objektvarianten automatisch platziert werden. \par 

Da das gemischte Trainingssets bessere Klassifikationsraten bietet, könnte es auch interessant sein, den Einfluss des Verhältnisses von echten zu unechten Bildern zu untersuchen. Sollte schon ein sehr kleiner Teil an echten Bildern reichen, um die Performance zu steigern, wäre dies ein Grund, weiterhin reale Bilder für Trainingszwecke aufzunehmen. \par 

Eine weitere Möglichkeit wäre es, die Klassifikationsgüte in ungeordneten Szenen zu untersuchen. Die für diese Arbeit erstellten synthetischen Bilder sind geordnet, das heißt die Objekte räumlich klar voneinander getrennt. Allerdings sind menschliche Umgebungen selten in diesem Sinne \textit{geordnet}. Für die Wahrnehmung eines Roboters in Haushaltsumgebungen sollte deshalb überprüft werden, ob sich synthetische Bilder auch für Objekterkennung in ungeordneten Bildern eignen. \par 

Es könnte auch versucht werden, die Instanzklassifikation zu verbessern.  Im menschlichen Haushalt ist es oftmals von Bedeutung, die Instanzen zu unterscheiden statt die Klassen. Wird zum Beispiel laktosefreie Milch mit der normalen Milch verwechselt, kann das für Menschen mit Laktoseintoleranz unangenehme Folgen haben. Um das zu vermeiden und einen autonomen Roboter in einem solchen Haushalt einsetzen zu können, sollte eine robustere Instanzklassifikation angestrebt sein. 