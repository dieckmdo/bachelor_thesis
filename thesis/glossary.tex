\newglossaryentry{fotoreal}
{
  name=Fotorealistisches Rendering,
  description={bullshit}
}
\newglossaryentry{gameengine}
{
  name=Game Engine,
  description={Komplexes und mächtiges Softwarepakt für die Erstellung und Entwicklung von Videospielen, sowie deren Ausführung.}
}
\newglossaryentry{framework}
{
  name=Framework,
  description={Stellt einen domänenspezifisches Programmiergerüst für die Anwendungsentwicklung zur Verfügung. Neben der Softwarearchitektur kann das auch den Kontrollfluss und Schnittstellen einschließen.}
}
\newglossaryentry{mn}
{
  name = {Markov Netzwerk},
  description={Eine ungerichtete Graphstruktur, die eine Wahrscheinlichkeitsverteilung von komplexen Zusammenhängen von Zufallsvariablen modelliert.}
}
\newacronym{engine}{Engine}{als Kurzform von Unreal Engine verwendet}
\newacronym{ros}{\textsc{ROS}}{\textsc{Robot Operating System}}
\newacronym{vfh}{VFH}{Viewpoint Feature Histogram}
\newacronym{cnn}{CNN}{Convolutional Neural Network}
\newacronym{iai}{IAI}{Institute for Artificial Intelligence}
\newacronym[plural={SofAs}]{sofa}{SofA}{Subject of Analysis}
\newacronym{cas}{CAS}{Common Analysis Structure}
\newacronym[plural={AEs},longplural={Analysis Engines}]{ae}{AE}{Analysis Engine}
\newacronym{rf}{RF}{Random Forest}
\newacronym{svm}{SVM}{Support Vector Machine}
\newacronym{knn}{KNN}{K-Nearest Neighbour}
\newacronym{bson}{BSON}{Binary JSON}
\newacronym{sql}{SQL}{Structured Query Language}
\newacronym{rgb}{RGB}{Rot-Grün-Blau}
\newacronym[plural={MLNs},longplural={Markov Logic Networks}]{mln}{MLN}{Markov Logic Network}
\newacronym{fo}{FO}{Prädikatenlogik erster Stufe}
\newacronym[plural={PGMs},longplural={Probabilistischen Graphischen Modellen}]{pgm}{PGM}{Probabilistische Graphische Modelle}
\newacronym{kb}{KB}{Wissensbasis}