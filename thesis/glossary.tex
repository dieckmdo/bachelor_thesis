\newglossaryentry{feature}
{
  name=Merkmale,
  plural = {Merkmalen},
  description={In der Computer Vision beschreiben Merkmale (engl. features) Informationen oder Strukturen in Bildern. Dies können unter anderem Punkte, Kanten, Formen oder ganze Objekte sein.}
}
\newglossaryentry{gameengine}
{
  name=Game Engine,
  plural = {Game Engines},
  description={Komplexes und mächtiges Softwarepakt für die Erstellung und Entwicklung von Videospielen sowie deren Ausführung.}
}
\newglossaryentry{klassifikator}
{
  name=Klassifikator,
  plural = {Klassifikatoren},
  description={Ein Klassifikator ordnet Objekte Klassen oder Kategorien zu. Im Rahmen der Robotik wird damit ein Algorithmus oder Programm beschrieben, das wahrgenommene Objekte erkennt, indem sie auf Basis verschiedener Eigenschaften einer bekannten Klasse zugeordnet werden.}
}
\newglossaryentry{framework}
{
  name=Framework,
  plural = {Frameworks},
  description={Stellt ein domänenspezifisches Programmiergerüst für die Anwendungsentwicklung zur Verfügung. Neben der Softwarearchitektur kann das auch den Kontrollfluss und Schnittstellen einschließen.}
}
\newglossaryentry{mn}
{
  name = {Markov Netzwerk},
  description={Eine ungerichtete Graphstruktur, die eine Wahrscheinlichkeitsverteilung von komplexen Zusammenhängen von Zufallsvariablen modelliert.}
}
\newglossaryentry{gt}
{
  name = {Ground Truth},
  description={Die Ground Truth beschreibt die Realität oder Wahrheit der wahrgenommenen Umgebung eines Roboters. Im Rahmen dieser Arbeit ist damit die tatsächliche Objektklasse oder Instanz eines Objektes gemeint.}
}
\newglossaryentry{accuracy}
{
  name = {Accuracy},
  description={Die Accuracy (dt.: Treffergenauigkeit) gibt an, wie viele Objekte insgesamt korrekt klassifiziert wurden. Sie entspricht \begin{displaymath}
P(korrekt\ klassifziert) = \frac{korrekt\ klassfiziert}{alle\ Fälle} 
\end{displaymath}}
}
\newglossaryentry{precision}
{
  name = {Precision},
  description={Die Precision (dt.: Relevanz oder Genauigkeit) beschreibt den Anteil der korrekten Klassifikationen aus den positiven Vorhersagen. Intuitiv ist das die Fähigkeit, eine negative Klassifikation nicht als positiv vorherzusagen. Sie entspricht der geschätzten bedingten Wahrscheinlichkeit \begin{displaymath}
P(korrekt\ klassifziert \mid positive\ Vorhersage) = \frac{richtig\ positiv}{richtig\ positiv + falsch\ positiv}
\end{displaymath}}
}
\newglossaryentry{recall}
{
  name = {Recall},
  description={Der Recall Wert (dt.: Trefferquote oder Sensitivität) beschreibt den Anteil der positiv korrekten Klassifikation aus den korrekten Klassifikationen. Sie entspricht der geschätzten bedingten Wahrscheinlichkeit \begin{displaymath}
P(positive\ Vorhersage \mid korrekt\ klassifziert) = \frac{richtig\ positiv}{richtig\ positiv + falsch\ negativ}
\end{displaymath}}
}
\newglossaryentry{f1score}
{
  name = {$F_1$-Maß},
  description={Das $F_1$-Maß ist das gewichtete Mittel aus \gls{precision} und \gls{recall}. Die Gewichtung beider Werte ist gleich. Berechnet wird das $F_1$-Maß durch
\begin{displaymath}
F_1 = 2 \times (Relevanz \cdot Trefferquote) : (Relevanz + Trefferquote)
\end{displaymath}}
}

\newacronym{engine}{Engine}{als Kurzform von Unreal Engine verwendet}
\newacronym{ros}{\textsc{ROS}}{\textsc{Robot Operating System}}
\newacronym{vfh}{VFH}{Viewpoint Feature Histogram}
\newacronym{cnn}{CNN}{Convolutional Neural Network}
\newacronym{iai}{IAI}{Institute for Artificial Intelligence}
\newacronym[plural={SofAs}]{sofa}{SofA}{Subject of Analysis}
\newacronym{cas}{CAS}{Common Analysis Structure}
\newacronym[plural={AEs},longplural={Analysis Engines}]{ae}{AE}{Analysis Engine}
\newacronym{rf}{RF}{Random Forest}
\newacronym{svm}{SVM}{Support Vector Machine}
\newacronym{knn}{KNN}{K-Nearest Neighbour}
\newacronym{bson}{BSON}{Binary JSON}
\newacronym{sql}{SQL}{Structured Query Language}
\newacronym{rgb}{RGB}{Rot-Grün-Blau}
\newacronym[plural={MLNs},longplural={Markov-Logik-Netzwerke}]{mln}{MLN}{Markov-Logik-Netzwerk}
\newacronym{mlns}{MLN}{Markov-Logik-Netzwerkes}
\newacronym{fo}{FO}{Prädikatenlogik erster Stufe}
\newacronym[plural={PGMs},longplural={Probabilistischen Graphischen Modellen}]{pgm}{PGM}{Probabilistische Graphische Modelle}
\newacronym[plural={KBs},longplural={Wissensbasen}]{kb}{KB}{Wissensbasis}
\newacronym{cad}{CAD}{Computer-aided design}
\newacronym{pcl}{PCL}{Point Cloud Library}
\newacronym[longplural={Virtuellen Realität}]{vr}{VR}{Virtuelle Realität}