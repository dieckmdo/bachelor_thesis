% ------------------------------------------------------------------------
%	Die KOMA-Script Dokumentklasse 'scrreprt' verwenden.
% http://www.dml.drzoom.ch/
% ------------------------------------------------------------------------
\documentclass[	pdftex, 
								a4paper,
								11pt, DIV11, BCOR5mm,
								parskip,								
								%draft,										
								%chapterprefix,					% entfernen	
								%pointlessnumbers					
								]{scrbook}
%
\makeatletter %%%%%%%%%%%%%%%%%%%%				% @ zu einem TeX-Zeichen machen
%
% ------------------------------------------------------------------------
%	Seitenlayout
% ------------------------------------------------------------------------
%\usepackage{setspace}
%\onehalfspace        										
%
\setcounter{secnumdepth}{4} 
\setcounter{tocdepth}{1}
%
\clubpenalty = 10000
\widowpenalty = 10000
\displaywidowpenalty = 10000
\usepackage{url}
\usepackage{amsmath}
%\usepackage{amssymb}
%\usepackage{amsfonts} 	

								
\usepackage{bbm}
% ------------------------------------------------------------------------
%	Schrift von serifenloser in Schrift *mit* Serifen ndern
% ------------------------------------------------------------------------
% alle auf einmal
%\setkomafont{sectioning}{\normalfont\bfseries}
%
%	Oder detailierter
%



%\setkomafont{chapter}{\normalfont\bfseries\sffamily\huge\color{chaptercolor}}				% fr Chapter
%\setkomafont{section}{\normalfont\bfseries\Large\color{chaptercolor}}			% fr Section
%\setkomafont{subsection}{\normalfont\bfseries\large\color{chaptercolor}}		% fr Subsection
%\setkomafont{subsubsection}{\normalfont\bfseries\color{chaptercolor}}			% fr SubSubSection
%\setkomafont{paragraph}{\normalfont\bfseries}					% fr Paragraph
%\setkomafont{subparagraph}{\normalfont\bfseries}				% fr SubParagraph
%\setkomafont{captionlabel}{\normalfont\bfseries\small} % fr Caption (Bildunterschrift usw.)
%


%\usepackage{times}

% ------------------------------------------------------------------------
%	Einstellen der deutschen Sprache, Verwendung von Umlauten, etc.
% ------------------------------------------------------------------------
%\usepackage[ngerman]{babel}							% neue deutsche Rechtschreibung
\usepackage[utf8]{inputenc}						% Eingabe von Umlauten ermglichen
%\usepackage[T1]{fontenc}								% Auswahl der Schrift
%\usepackage{ae}
\nonfrenchspacing 											% Nach Satzende groesserer Abstand
%
% ------------------------------------------------------------------------
%	Kopfzeilen
% http://www.physik.tu-berlin.de/pcpool/phcd/Dokumentation/LatexDoku/Proseminar/V07-footnote.pdf
% ftp://ftp.ctan.org/tex-archive/macros/latex/contrib/koma-script/scrguide.pdf (nach chaptermarkformat suchen)
% ------------------------------------------------------------------------
% 
\usepackage[automark,headsepline,headinclude]{scrpage2}

            
	% Kapitelnummer vor Kapiteltitel stellen
%\renewcommand*{\chaptermarkformat}{}			% nur Kapitel ohne 'Kapitel' oder Nummer anzeigen
%\renewcommand*{\sectionmarkformat}{}			% nur Section ohne Nummer anzeigen


\pagestyle{scrheadings}
\clearscrheadings													% Voreinstellugnen lschen
\clearscrplain														% Voreinstellugnen lschen
%\ohead{\leftmark\ -\ \rightmark}				% oben links: 'Kapitel - Section'
\ihead{\leftmark}													% oben links: 'Kapitel'
\ohead{\thepage}												% oben rechts aktuelle Seite
%\cfoot{\thepage}													
\setheadsepline{.4pt} 										% Linie unter der Kopfzeile
%\setfootsepline{.4pt}
\setlength{\headheight}{1.1\baselineskip}	% bei 1,5-fachem Zeilenabstand

%
% ------------------------------------------------------------------------
%	Glossar
% http://mrunix.de/forums/showthread.php?t=39944
%	http://www.dante.de/CTAN/macros/latex/contrib/nomencl/nomencl.pdf
% ------------------------------------------------------------------------
%
%\usepackage[german]{nomencl}							% nomenclature Glossar verwenden
%\usepackage{nomencl}
%\renewcommand{\nomname}{Glossary}					% Symbolverzeichnis in Glossar umbenennen
%\setlength{\nomlabelwidth}{.25\hsize}			% Breite des Begriffs (hier: 25% der Gesamtbreite)
%\renewcommand{\nomlabel}[1]{#1 \dotfill}	% Abstand zwischen Begriff und Beschreibung mit Punkten fllen
%\setlength{\nomitemsep}{-\parsep}					% Abstand der Linien zueinander
%
%\makenomenclature													% Glossar erzeugen
%\newglossaryentry{fotoreal}
{
  name=Fotorealistisches Rendering,
  description={bullshit}
}
\newglossaryentry{gameengine}
{
  name=Game Engine,
  plural = {Game Engines},
  description={Komplexes und mächtiges Softwarepakt für die Erstellung und Entwicklung von Videospielen, sowie deren Ausführung.}
}
\newglossaryentry{framework}
{
  name=Framework,
  description={Stellt einen domänenspezifisches Programmiergerüst für die Anwendungsentwicklung zur Verfügung. Neben der Softwarearchitektur kann das auch den Kontrollfluss und Schnittstellen einschließen.}
}
\newglossaryentry{mn}
{
  name = {Markov Netzwerk},
  description={Eine ungerichtete Graphstruktur, die eine Wahrscheinlichkeitsverteilung von komplexen Zusammenhängen von Zufallsvariablen modelliert.}
}
\newacronym{engine}{Engine}{als Kurzform von Unreal Engine verwendet}
\newacronym{ros}{\textsc{ROS}}{\textsc{Robot Operating System}}
\newacronym{vfh}{VFH}{Viewpoint Feature Histogram}
\newacronym{cnn}{CNN}{Convolutional Neural Network}
\newacronym{iai}{IAI}{Institute for Artificial Intelligence}
\newacronym[plural={SofAs}]{sofa}{SofA}{Subject of Analysis}
\newacronym{cas}{CAS}{Common Analysis Structure}
\newacronym[plural={AEs},longplural={Analysis Engines}]{ae}{AE}{Analysis Engine}
\newacronym{rf}{RF}{Random Forest}
\newacronym{svm}{SVM}{Support Vector Machine}
\newacronym{knn}{KNN}{K-Nearest Neighbour}
\newacronym{bson}{BSON}{Binary JSON}
\newacronym{sql}{SQL}{Structured Query Language}
\newacronym{rgb}{RGB}{Rot-Grün-Blau}
\newacronym[plural={MLNs},longplural={Markov Logic Networks}]{mln}{MLN}{Markov Logic Network}
\newacronym{fo}{FO}{Prädikatenlogik erster Stufe}
\newacronym[plural={PGMs},longplural={Probabilistischen Graphischen Modellen}]{pgm}{PGM}{Probabilistische Graphische Modelle}
\newacronym[plural={KBs},longplural={Wissensbasen}]{kb}{KB}{Wissensbasis}									% Glossareintrge aus extra Datei lesen
%
% ------------------------------------------------------------------------
%	Literaturverzeichnis
% ------------------------------------------------------------------------
%
%\usepackage{bibgerm}											% deutsche Anpassungen ans Literaturverzeichnis
%
%
% ------------------------------------------------------------------------
%	Stichwortverzeichnis
% http://www2.informatik.hu-berlin.de/~goerlach/Indexerstellung.pdf
% ------------------------------------------------------------------------
%
%\usepackage{makeidx}
%\makeindex																% Stichwortverzeichnis erzeugen
%
% ------------------------------------------------------------------------
%	PDF-Einstellungen
% http://www.math.uni-hamburg.de/home/iffland/Materialien/Einf_hyperref.pdf
% ------------------------------------------------------------------------

%\usepackage{lmodern}
\usepackage[pdfpagelabels,
						bookmarksopen,
						pdfstartview={FitV},
						plainpages=false,
						breaklinks=true,
						linkbordercolor={1 1 1},
						citebordercolor={1 1 1},
						urlbordercolor={1 1 1}]{hyperref}  % Stellt Verlinkungen im Dokument her
% Beim mehrmaligen vergeben von Seitenummern (rmisch,arabisch) gibt es gelegentlich
% Warnungen. Diese knnen aber ignoriert werden. siehe hierzu: 
% http://www.tex.ac.uk/cgi-bin/texfaq2html?label=hyperdupdest
%


% ------------------------------------------------------------------------
%	Grafics
% ------------------------------------------------------------------------
%\usepackage{graphicx}
%\usepackage[rm,bf,it,RM,IT]{subfigure}
\usepackage[usenames,dvipsnames]{color}
\definecolor{mblue}{gray}{0.5} % For some reason, this color is needed for sections			
%
% ------------------------------------------------------------------------
%	extended Tables
% ------------------------------------------------------------------------
%\usepackage{tabularx}
%\usepackage{booktabs}
%\usepackage{multirow}
%\usepackage{multicol}
%
% ------------------------------------------------------------------------
%	Font Settings
%  Available: Font Styles: cmr, guc; phv , pbk, pzc, ptm, ppl, pag
% ------------------------------------------------------------------------
% nonfree: ul9, ulg, ual
%\usepackage{kpfonts}
%\usepackage{lmodern}
%\usepackage{times}
%\usepackage{iwona}
%\usepackage{arev}
%\usepackage{cmbright}
\usepackage[charter]{mathdesign}
%\renewcommand{\sfdefault}{cmbr} %fvs
%\renewcommand{\rmdefault}{cmr}%{ppl}


\definecolor{chaptercolor}{rgb}{0,.639,.827}
\definecolor{sectioncolor}{rgb}{0,.639,.827}
%\definecolor{chapterbar}{rgb}{.56,.8,.56}%{.4,.639,.827}
\definecolor{chapterbar}{rgb}{.84,.09,.23}%AI Color

%\addtokomafont{chapter}{\usefont{T1}{fav}{b}{it}\color{black}}
\addtokomafont{section}{\usefont{T1}{fav}{b}{}\color{black}}
\addtokomafont{subsection}{\usefont{T1}{fav}{b}{}\color{black}}
\addtokomafont{subsubsection}{\usefont{T1}{fav}{b}{}\color{black}}

\addtokomafont{captionlabel}{\usefont{T1}{ppl}{b}{it}}%lmss
\addtokomafont{caption}{\usefont{T1}{ppl}{m}{it}}

\addtokomafont{sectioning}{\usefont{T1}{fav}{b}{it}}

\renewcommand{\headfont}{\usefont{T1}{lmss}{m}{it}}%uop

\usepackage{titlesec,xcolor,lipsum}
\automark[section]{chapter}

\setlength{\unitlength}{1cm}
\titleformat{\chapter}[display]
    { \usefont{T1}{fav}{b}{it}}
    { \hspace{1.5cm} \large{\chaptertitlename} \thechapter} 
    { .1pc }
    { \begin{picture}(10,0)(0,0)\color{chapterbar}\linethickness{.25cm}\put(.5,-.3){\line(0,1){2}} \hspace{1.5cm} \color{black}\Huge\usefont{T1}{fav}{b}{}}
    [ \end{picture} ]

%\usepackage{shadethm}

%\newshadetheorem{thms}{Theorem}[chapter]

%\newenvironment{thm}[1][]{%
%  \definecolor{shadethmcolor}{rgb}{.929,.968,.992}%
 % \definecolor{shaderulecolor}{rgb}{0.0,0.0,0.4}%
 % \setlength{\shadeboxrule}{1.5pt}%
%  \usefont{T1}{lmss}{m}{it}
%  \begin{thms}[#1]\hspace*{1mm}%
%}{\end{thms}}


%
% ------------------------------------------------------------------------
%	Erweiterung um MetaPost-Dateien direkt einzubinden.
% ------------------------------------------------------------------------
%\makeatletter
%\@ifundefined{pdfoutput}{}{\DeclareGraphicsRule{*}{mps}{*}{}}
%\makeatother
%
% ------------------------------------------------------------------------
%	Fu�zeilen in \caption einer float Umgebung verwenden.
% http://www.tex.ac.uk/tex-archive/macros/latex/contrib/ccaption/ccaption.pdf
% ------------------------------------------------------------------------
\usepackage[caption2]{ccaption}
\usepackage{dsfont}

\newenvironment{items}{\begin{itemize}\setlength{\itemsep}{-5mm} \setlength{\topsep}{-10mm} \setlength{\parsep}{20mm}}{\end{itemize}} 

\newenvironment{enum}{\begin{enumerate}\setlength{\itemsep}{-5mm} \setlength{\topsep}{-5mm} \setlength{\parsep}{-5mm}}{\end{enumerate}}
